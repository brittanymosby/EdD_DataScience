% Options for packages loaded elsewhere
\PassOptionsToPackage{unicode}{hyperref}
\PassOptionsToPackage{hyphens}{url}
%
\documentclass[
]{article}
\usepackage{lmodern}
\usepackage{amssymb,amsmath}
\usepackage{ifxetex,ifluatex}
\ifnum 0\ifxetex 1\fi\ifluatex 1\fi=0 % if pdftex
  \usepackage[T1]{fontenc}
  \usepackage[utf8]{inputenc}
  \usepackage{textcomp} % provide euro and other symbols
\else % if luatex or xetex
  \usepackage{unicode-math}
  \defaultfontfeatures{Scale=MatchLowercase}
  \defaultfontfeatures[\rmfamily]{Ligatures=TeX,Scale=1}
\fi
% Use upquote if available, for straight quotes in verbatim environments
\IfFileExists{upquote.sty}{\usepackage{upquote}}{}
\IfFileExists{microtype.sty}{% use microtype if available
  \usepackage[]{microtype}
  \UseMicrotypeSet[protrusion]{basicmath} % disable protrusion for tt fonts
}{}
\makeatletter
\@ifundefined{KOMAClassName}{% if non-KOMA class
  \IfFileExists{parskip.sty}{%
    \usepackage{parskip}
  }{% else
    \setlength{\parindent}{0pt}
    \setlength{\parskip}{6pt plus 2pt minus 1pt}}
}{% if KOMA class
  \KOMAoptions{parskip=half}}
\makeatother
\usepackage{xcolor}
\IfFileExists{xurl.sty}{\usepackage{xurl}}{} % add URL line breaks if available
\IfFileExists{bookmark.sty}{\usepackage{bookmark}}{\usepackage{hyperref}}
\hypersetup{
  hidelinks,
  pdfcreator={LaTeX via pandoc}}
\urlstyle{same} % disable monospaced font for URLs
\usepackage[margin=1in]{geometry}
\usepackage{color}
\usepackage{fancyvrb}
\newcommand{\VerbBar}{|}
\newcommand{\VERB}{\Verb[commandchars=\\\{\}]}
\DefineVerbatimEnvironment{Highlighting}{Verbatim}{commandchars=\\\{\}}
% Add ',fontsize=\small' for more characters per line
\usepackage{framed}
\definecolor{shadecolor}{RGB}{248,248,248}
\newenvironment{Shaded}{\begin{snugshade}}{\end{snugshade}}
\newcommand{\AlertTok}[1]{\textcolor[rgb]{0.94,0.16,0.16}{#1}}
\newcommand{\AnnotationTok}[1]{\textcolor[rgb]{0.56,0.35,0.01}{\textbf{\textit{#1}}}}
\newcommand{\AttributeTok}[1]{\textcolor[rgb]{0.77,0.63,0.00}{#1}}
\newcommand{\BaseNTok}[1]{\textcolor[rgb]{0.00,0.00,0.81}{#1}}
\newcommand{\BuiltInTok}[1]{#1}
\newcommand{\CharTok}[1]{\textcolor[rgb]{0.31,0.60,0.02}{#1}}
\newcommand{\CommentTok}[1]{\textcolor[rgb]{0.56,0.35,0.01}{\textit{#1}}}
\newcommand{\CommentVarTok}[1]{\textcolor[rgb]{0.56,0.35,0.01}{\textbf{\textit{#1}}}}
\newcommand{\ConstantTok}[1]{\textcolor[rgb]{0.00,0.00,0.00}{#1}}
\newcommand{\ControlFlowTok}[1]{\textcolor[rgb]{0.13,0.29,0.53}{\textbf{#1}}}
\newcommand{\DataTypeTok}[1]{\textcolor[rgb]{0.13,0.29,0.53}{#1}}
\newcommand{\DecValTok}[1]{\textcolor[rgb]{0.00,0.00,0.81}{#1}}
\newcommand{\DocumentationTok}[1]{\textcolor[rgb]{0.56,0.35,0.01}{\textbf{\textit{#1}}}}
\newcommand{\ErrorTok}[1]{\textcolor[rgb]{0.64,0.00,0.00}{\textbf{#1}}}
\newcommand{\ExtensionTok}[1]{#1}
\newcommand{\FloatTok}[1]{\textcolor[rgb]{0.00,0.00,0.81}{#1}}
\newcommand{\FunctionTok}[1]{\textcolor[rgb]{0.00,0.00,0.00}{#1}}
\newcommand{\ImportTok}[1]{#1}
\newcommand{\InformationTok}[1]{\textcolor[rgb]{0.56,0.35,0.01}{\textbf{\textit{#1}}}}
\newcommand{\KeywordTok}[1]{\textcolor[rgb]{0.13,0.29,0.53}{\textbf{#1}}}
\newcommand{\NormalTok}[1]{#1}
\newcommand{\OperatorTok}[1]{\textcolor[rgb]{0.81,0.36,0.00}{\textbf{#1}}}
\newcommand{\OtherTok}[1]{\textcolor[rgb]{0.56,0.35,0.01}{#1}}
\newcommand{\PreprocessorTok}[1]{\textcolor[rgb]{0.56,0.35,0.01}{\textit{#1}}}
\newcommand{\RegionMarkerTok}[1]{#1}
\newcommand{\SpecialCharTok}[1]{\textcolor[rgb]{0.00,0.00,0.00}{#1}}
\newcommand{\SpecialStringTok}[1]{\textcolor[rgb]{0.31,0.60,0.02}{#1}}
\newcommand{\StringTok}[1]{\textcolor[rgb]{0.31,0.60,0.02}{#1}}
\newcommand{\VariableTok}[1]{\textcolor[rgb]{0.00,0.00,0.00}{#1}}
\newcommand{\VerbatimStringTok}[1]{\textcolor[rgb]{0.31,0.60,0.02}{#1}}
\newcommand{\WarningTok}[1]{\textcolor[rgb]{0.56,0.35,0.01}{\textbf{\textit{#1}}}}
\usepackage{graphicx,grffile}
\makeatletter
\def\maxwidth{\ifdim\Gin@nat@width>\linewidth\linewidth\else\Gin@nat@width\fi}
\def\maxheight{\ifdim\Gin@nat@height>\textheight\textheight\else\Gin@nat@height\fi}
\makeatother
% Scale images if necessary, so that they will not overflow the page
% margins by default, and it is still possible to overwrite the defaults
% using explicit options in \includegraphics[width, height, ...]{}
\setkeys{Gin}{width=\maxwidth,height=\maxheight,keepaspectratio}
% Set default figure placement to htbp
\makeatletter
\def\fps@figure{htbp}
\makeatother
\setlength{\emergencystretch}{3em} % prevent overfull lines
\providecommand{\tightlist}{%
  \setlength{\itemsep}{0pt}\setlength{\parskip}{0pt}}
\setcounter{secnumdepth}{-\maxdimen} % remove section numbering

\author{}
\date{\vspace{-2.5em}}

\begin{document}

\hypertarget{intro}{%
\section{Welcome to Data Science!}\label{intro}}

Today, we'll be working on getting you set up with the tools you will
need for this class. Once you are set up, we'll do what we're here to
do: analyze data!

Here's what we need to get done today:

\begin{enumerate}
\def\labelenumi{\arabic{enumi}.}
\tightlist
\item
  Introductions\\
\item
  Installing R
\item
  Installing Rstudio
\item
  Learning the basic verbs of data analysis
\end{enumerate}

\hypertarget{introductions}{%
\subsection{Introductions}\label{introductions}}

We need two basic sets of tools for this class. We will need \texttt{R}
to analyze data. We will need \texttt{RStudio} to help us interface with
R and to produce documentation of our results.

\hypertarget{installing-r}{%
\subsection{Installing R}\label{installing-r}}

R is going to be the only programming language we will use. R is an
extensible statistical programming environment that can handle all of
the main tasks that we'll need to cover this semester: getting data,
analyzing data and communicating data analysis.

If you haven't already, you need to download R here:
\url{https://cran.r-project.org/}.

\hypertarget{installing-rstudio}{%
\subsection{Installing Rstudio}\label{installing-rstudio}}

When we work with R, we communicate via the command line. To help
automate this process, we can write scripts, which contain all of the
commands to be executed. These scripts generate various kinds of output,
like numbers on the screen, graphics or reports in common formats (pdf,
word). Most programming languages have several \textbf{I} ntegrated
\textbf{D} evelopment \textbf{E} nvironments (IDEs) that encompass all
of these elements (scripts, command line interface, output). The primary
IDE for R is Rstudio.

If you haven't already, you need to download Rstudio here:
\url{https://www.rstudio.com/products/rstudio/download2/}. You need the
free Rstudio desktop version.

\hypertarget{rmd-files}{%
\subsection{.Rmd files}\label{rmd-files}}

Open the \texttt{01-Intro.Rmd} file. In Rstudio, go to
File--\textgreater Open, then find the \texttt{01-Intro.Rmd} file in the
directory.

.Rmd files will be the only file format we work in this class. .Rmd
files contain three basic elements:

\begin{enumerate}
\def\labelenumi{\arabic{enumi}.}
\tightlist
\item
  Script that can be interpreted by R.
\item
  Output generated by R, including tables and figures.\\
\item
  Text that can be read by humans.
\end{enumerate}

From a .Rmd file you can generate html documents, pdf documents, word
documents, slides . . . lots of stuff. All class notes will be in .Rmd.
All assignments will be turned in as .Rmd files, and your final project?
You guessed it, .Rmd.

In the \texttt{01-Intro.Rmd} file you'll notice that there are three
open single quotes in a row, like so:
\texttt{\textasciigrave{}\textasciigrave{}\textasciigrave{}} This
indicates the start of a ``code chunk'' in our file. The first code
chunk that we load will include a set of programs that we will need all
semester long.

\hypertarget{r-packages}{%
\subsection{R Packages}\label{r-packages}}

When we say that R is extensible, we mean that people in the community
can write programs that everyone else can use. These are called
``packages.'' In these first few lines of code, I load a set of packages
using the library command in R. The set of packages, called
\texttt{tidyverse} were written by Hadley Wickham and play a key role in
his book. To install this set of packages, simply type in
\texttt{install.packages("tidyverse")} at the R command prompt.

To run the code below in R, you can:

\begin{itemize}
\tightlist
\item
  Press the ``play'' button next to the code chunk
\item
  In OS X, place the cursor in the code chunk and hit
  \texttt{cmd+shift+enter}
\item
  In Windows, place the cursor in the code chunk and hit
  \texttt{ctrl+shefit+enter}
\end{itemize}

\hypertarget{this-section-prints-the-number-4}{%
\section{This section prints the number
4}\label{this-section-prints-the-number-4}}

\begin{Shaded}
\begin{Highlighting}[]
\DecValTok{2}\OperatorTok{+}\DecValTok{2}
\end{Highlighting}
\end{Shaded}

\begin{verbatim}
## [1] 4
\end{verbatim}

\begin{Shaded}
\begin{Highlighting}[]
\CommentTok{## Clear environment}
\KeywordTok{rm}\NormalTok{(}\DataTypeTok{list=}\KeywordTok{ls}\NormalTok{())}
\CommentTok{## Get necessary libraries-- won't work the first time, because you need to install them!}
\KeywordTok{library}\NormalTok{(tidyverse)}
\end{Highlighting}
\end{Shaded}

\begin{verbatim}
## -- Attaching packages -------------------------------------------------------------------------------------------- tidyverse 1.3.0 --
\end{verbatim}

\begin{verbatim}
## v ggplot2 3.3.2     v purrr   0.3.4
## v tibble  3.0.3     v dplyr   1.0.2
## v tidyr   1.1.2     v stringr 1.4.0
## v readr   1.4.0     v forcats 0.5.0
\end{verbatim}

\begin{verbatim}
## -- Conflicts ----------------------------------------------------------------------------------------------- tidyverse_conflicts() --
## x dplyr::filter() masks stats::filter()
## x dplyr::lag()    masks stats::lag()
\end{verbatim}

Now we're ready to load in data. The data frame will be our basic way of
interacting with everything in this class. The \texttt{sc} data frame
contains information from the college scorecard on 127 different
colleges and univeristies.

However, we first need to make sure that R is looking in the right
place. When you opened up your project, Rstudio automagically took you
to the directory for that project. But because we keep lessons in a
separate directory, we need to

\begin{Shaded}
\begin{Highlighting}[]
\CommentTok{## Load in the data}
\KeywordTok{load}\NormalTok{(}\StringTok{"college.Rdata"}\NormalTok{)}
\end{Highlighting}
\end{Shaded}

Here are the variables in the \texttt{college.Rdata} dataset:

\emph{Variable Name} :\emph{Definition} unitid: Unit ID

instnm: Institution Name

stabbr: State Abbreviation

year: Year

control: control of institution, 1=public, 2= private non-profit,
3=private for-profit

preddeg: predominant degree, 1= certificate, 2= associates, 3=
bachelor's, 4=graduate

adm\_rate: Proportion of Applicants Admitted

sat\_avg: Midpoint of entrance exam scores, on SAT scale, math and
verbal only

costt\_4a: Average cost of attendance (tuition and room and board less
all grant aid)

debt\_mdn: Median debt of graduates

md\_earn\_ne\_pg: Earnings of graduates who are not enrolled in higher
education, six years after graduation

\emph{Looking at datasets}

We can look at the first few rows and columns of \texttt{sc} by typing
in the data name.

We can look at the whole dataset using View.

\begin{Shaded}
\begin{Highlighting}[]
\CommentTok{## What does this data look like? Look at the first few rows, first few variables}
\NormalTok{sc}
\end{Highlighting}
\end{Shaded}

\begin{verbatim}
## # A tibble: 125 x 12
##    unitid instnm stabbr  year control preddeg adm_rate sat_avg costt4_a debt_mdn
##     <int> <chr>  <chr>  <dbl>   <int>   <int>    <dbl>   <dbl>    <int>    <dbl>
##  1 446048 Ave M~ FL      2009       2       3    0.374    1104    29200    8875 
##  2 443410 DigiP~ WA      2009       3       3    0.326    1194    23969   16125 
##  3 111081 Calif~ CA      2009       2       3    0.283      NA    48784   18188.
##  4 112260 Clare~ CA      2009       2       3    0.163    1389    50990    6500 
##  5 113537 Dell'~ CA      2009       2       1    0          NA       NA    9500 
##  6 404338 Schil~ FL      2009       3       2    0.158      NA    35408    6500 
##  7 117928 Argos~ CA      2009       3       3    0.323      NA    35858    9616.
##  8 120537 Hope ~ CA      2009       2       3    0.386     975    33366   10250 
##  9 119544 The N~ CA      2009       3       3    0.261      NA    19135    5500 
## 10 107071 Hende~ AR      2009       1       3    0.313    1048    14629    7500 
## # ... with 115 more rows, and 2 more variables: md_earn_wne_p6 <int>,
## #   ugds <int>
\end{verbatim}

\begin{Shaded}
\begin{Highlighting}[]
\CommentTok{## This is commented as you might not always want to run it. Delete the "#" sign below in order to run the command. }

\KeywordTok{View}\NormalTok{(sc)}
\end{Highlighting}
\end{Shaded}

\emph{Filter, Select, Arrange}

In exploring data, many times we want to look at smaller parts of the
dataset. There are three commands we'll use today that help with this.

-\texttt{filter} selects only those cases or rows that meet some logical
criteria.

-\texttt{select} selects only those variables or columns that meet some
criteria

-\texttt{arrange} arranges the rows of a dataset in the way we want.

For more on these, please see this
\href{https://cran.rstudio.com/web/packages/dplyr/vignettes/introduction.html}{vignette}.

Let's grab just the data for Vanderbilt, then look only at the average
test scores and admit rate.

\begin{Shaded}
\begin{Highlighting}[]
\CommentTok{## Where are we?}
\NormalTok{sc}\OperatorTok\KeywordTok{filter}\NormalTok{(instnm}\OperatorTok{==}\StringTok{"Vanderbilt University"}\NormalTok{)}
\end{Highlighting}
\end{Shaded}

\begin{verbatim}
## # A tibble: 1 x 12
##   unitid instnm stabbr  year control preddeg adm_rate sat_avg costt4_a debt_mdn
##    <int> <chr>  <chr>  <dbl>   <int>   <int>    <dbl>   <dbl>    <int>    <dbl>
## 1 221999 Vande~ TN      2009       2       3    0.202    1430    52303    12625
## # ... with 2 more variables: md_earn_wne_p6 <int>, ugds <int>
\end{verbatim}

\begin{Shaded}
\begin{Highlighting}[]
\NormalTok{sc}\OperatorTok\KeywordTok{filter}\NormalTok{(instnm}\OperatorTok{==}\StringTok{"Vanderbilt University"}\NormalTok{)}\OperatorTok\KeywordTok{select}\NormalTok{(instnm,adm_rate,sat_avg )}
\end{Highlighting}
\end{Shaded}

\begin{verbatim}
## # A tibble: 1 x 3
##   instnm                adm_rate sat_avg
##   <chr>                    <dbl>   <dbl>
## 1 Vanderbilt University    0.202    1430
\end{verbatim}

This is actually pretty powerful- we can quickly learn a lot just by
looking at individuals. We can also get groups of institutions by
specifying the characteristics of institutions we'd like to look at. For
instance, let's look at institutions with admission rates of less than
10 percent. We'll then select three characteristics: name, admit rate
and average SAT scores. Then we'll order the result by SAT scores.

\begin{Shaded}
\begin{Highlighting}[]
\CommentTok{## Just colleges with low admit rates: order by sat scores (- sat_avg gives descending)}
\NormalTok{sc}\OperatorTok\KeywordTok{filter}\NormalTok{(adm_rate}\OperatorTok{<}\NormalTok{.}\DecValTok{1}\NormalTok{)}\OperatorTok\KeywordTok{select}\NormalTok{(instnm,adm_rate,sat_avg)}\OperatorTok
\StringTok{  }\KeywordTok{arrange}\NormalTok{(}\OperatorTok{-}\NormalTok{sat_avg)}
\end{Highlighting}
\end{Shaded}

\begin{verbatim}
## # A tibble: 6 x 3
##   instnm                                              adm_rate sat_avg
##   <chr>                                                  <dbl>   <dbl>
## 1 Yale University                                       0.0856    1475
## 2 Harvard University                                    0.0719    1468
## 3 Stanford University                                   0.0797    1436
## 4 Cooper Union for the Advancement of Science and Art   0.0735    1336
## 5 Dell'Arte International School of Physical Theatre    0           NA
## 6 The Juilliard School                                  0.0711      NA
\end{verbatim}

\emph{Summarizing Data}

We can summarize data by giving commands to the summarize operator. For
instance, what if we wanted to know what average debt levels looked like
across these colleges?

\begin{Shaded}
\begin{Highlighting}[]
\CommentTok{## What's the average median debt?}
\NormalTok{sc}\OperatorTok\KeywordTok{summarize}\NormalTok{(}\DataTypeTok{mean_debt=}\KeywordTok{mean}\NormalTok{(debt_mdn,}\DataTypeTok{na.rm=}\OtherTok{TRUE}\NormalTok{))}
\end{Highlighting}
\end{Shaded}

\begin{verbatim}
## # A tibble: 1 x 1
##   mean_debt
##       <dbl>
## 1    11277.
\end{verbatim}

\emph{Combining Commands} We can also combine commands, so that
summaries are done on only a part of the dataset. Below, I summarize
median debt for selective schools, and not very selective schools.

\begin{Shaded}
\begin{Highlighting}[]
\CommentTok{## What's the average median debt for very selective schools?}
\NormalTok{sc}\OperatorTok\KeywordTok{filter}\NormalTok{(adm_rate}\OperatorTok{<}\NormalTok{.}\DecValTok{1}\NormalTok{)}\OperatorTok\KeywordTok{summarize}\NormalTok{(}\DataTypeTok{mean_debt=}\KeywordTok{mean}\NormalTok{(debt_mdn,}\DataTypeTok{na.rm=}\OtherTok{TRUE}\NormalTok{))}
\end{Highlighting}
\end{Shaded}

\begin{verbatim}
## # A tibble: 1 x 1
##   mean_debt
##       <dbl>
## 1     9336.
\end{verbatim}

\begin{Shaded}
\begin{Highlighting}[]
\CommentTok{## And for not very selective schools?}
\NormalTok{sc}\OperatorTok\KeywordTok{filter}\NormalTok{(adm_rate}\OperatorTok{>}\NormalTok{.}\DecValTok{3}\NormalTok{)}\OperatorTok\KeywordTok{summarize}\NormalTok{(}\DataTypeTok{mean_debt=}\KeywordTok{mean}\NormalTok{(debt_mdn,}\DataTypeTok{na.rm=}\OtherTok{TRUE}\NormalTok{))}
\end{Highlighting}
\end{Shaded}

\begin{verbatim}
## # A tibble: 1 x 1
##   mean_debt
##       <dbl>
## 1    11684.
\end{verbatim}

\emph{Quick Exercise: What is the admission rate at UNC Chapel Hill?}

\begin{Shaded}
\begin{Highlighting}[]
\NormalTok{sc}\OperatorTok\KeywordTok{filter}\NormalTok{(instnm}\OperatorTok{==}\StringTok{"University of North Carolina at Chapel Hill"}\NormalTok{)}
\end{Highlighting}
\end{Shaded}

\begin{verbatim}
## # A tibble: 1 x 12
##   unitid instnm stabbr  year control preddeg adm_rate sat_avg costt4_a debt_mdn
##    <int> <chr>  <chr>  <dbl>   <int>   <int>    <dbl>   <dbl>    <int>    <dbl>
## 1 199120 Unive~ NC      2009       1       3    0.340    1305    16194    11000
## # ... with 2 more variables: md_earn_wne_p6 <int>, ugds <int>
\end{verbatim}

\end{document}
